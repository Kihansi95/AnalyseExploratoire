\documentclass[11pt,]{article}
\usepackage{lmodern}
\usepackage{amssymb,amsmath}
\usepackage{ifxetex,ifluatex}
\usepackage{fixltx2e} % provides \textsubscript
\ifnum 0\ifxetex 1\fi\ifluatex 1\fi=0 % if pdftex
  \usepackage[T1]{fontenc}
  \usepackage[utf8]{inputenc}
\else % if luatex or xelatex
  \ifxetex
    \usepackage{mathspec}
  \else
    \usepackage{fontspec}
  \fi
  \defaultfontfeatures{Ligatures=TeX,Scale=MatchLowercase}
\fi
% use upquote if available, for straight quotes in verbatim environments
\IfFileExists{upquote.sty}{\usepackage{upquote}}{}
% use microtype if available
\IfFileExists{microtype.sty}{%
\usepackage{microtype}
\UseMicrotypeSet[protrusion]{basicmath} % disable protrusion for tt fonts
}{}
\usepackage[margin=1in]{geometry}
\usepackage{hyperref}
\hypersetup{unicode=true,
            pdftitle={Compte-Rendu du projet d'Analyse Exploratoire},
            pdfauthor={Duc Hau NGUYEN; Anaïs RABARY},
            pdfborder={0 0 0},
            breaklinks=true}
\urlstyle{same}  % don't use monospace font for urls
\usepackage{graphicx,grffile}
\makeatletter
\def\maxwidth{\ifdim\Gin@nat@width>\linewidth\linewidth\else\Gin@nat@width\fi}
\def\maxheight{\ifdim\Gin@nat@height>\textheight\textheight\else\Gin@nat@height\fi}
\makeatother
% Scale images if necessary, so that they will not overflow the page
% margins by default, and it is still possible to overwrite the defaults
% using explicit options in \includegraphics[width, height, ...]{}
\setkeys{Gin}{width=\maxwidth,height=\maxheight,keepaspectratio}
\IfFileExists{parskip.sty}{%
\usepackage{parskip}
}{% else
\setlength{\parindent}{0pt}
\setlength{\parskip}{6pt plus 2pt minus 1pt}
}
\setlength{\emergencystretch}{3em}  % prevent overfull lines
\providecommand{\tightlist}{%
  \setlength{\itemsep}{0pt}\setlength{\parskip}{0pt}}
\setcounter{secnumdepth}{0}
% Redefines (sub)paragraphs to behave more like sections
\ifx\paragraph\undefined\else
\let\oldparagraph\paragraph
\renewcommand{\paragraph}[1]{\oldparagraph{#1}\mbox{}}
\fi
\ifx\subparagraph\undefined\else
\let\oldsubparagraph\subparagraph
\renewcommand{\subparagraph}[1]{\oldsubparagraph{#1}\mbox{}}
\fi

%%% Use protect on footnotes to avoid problems with footnotes in titles
\let\rmarkdownfootnote\footnote%
\def\footnote{\protect\rmarkdownfootnote}

%%% Change title format to be more compact
\usepackage{titling}

% Create subtitle command for use in maketitle
\newcommand{\subtitle}[1]{
  \posttitle{
    \begin{center}\large#1\end{center}
    }
}

\setlength{\droptitle}{-2em}

  \title{Compte-Rendu du projet d'Analyse Exploratoire}
    \pretitle{\vspace{\droptitle}\centering\huge}
  \posttitle{\par}
  \subtitle{Des médailles aux JO !}
  \author{Duc Hau NGUYEN; Anaïs RABARY}
    \preauthor{\centering\large\emph}
  \postauthor{\par}
      \predate{\centering\large\emph}
  \postdate{\par}
    \date{4 Décembre 2018}


\begin{document}
\maketitle

{
\setcounter{tocdepth}{3}
\tableofcontents
}
\newpage

\section{Introduction}\label{introduction}

\subsection{Contexte}\label{contexte}

Les Jeux Olympiques (JO) sont une compétition dont nous avons hérité de
la Grèce Antique. On lit souvent que seuls les citoyens les plus riches
pouvaient y participer. La trève olympique permettait à la civilisation
d'oublier, le temps de 12 travaux, les guerres et troubles de l'époque.
Après quelques maigres tentatives, c'est seulement depuis 1896 que les
Jeux Olympiques modernes voient le jour, à Athènes.

Dans le cadre du projet d'Analyse Exploratoire à l'INSA Toulouse, avec
l'accompagnement de notre professeur Gilles Tredan, nous avons récupéré
sur le site Kaggle un dataset sur ces Jeux Olympiques moderne. Ce
dernier recense des informations sur tous les athlètes ayant participés
aux JO, les médailles qu'ils ont gagnés, leur équipe, l'année de
participation, etc. Suivant les axes explorés dans ce rapport et
détaillés ci-après, nous avons eu besoin de croiser nos données avec des
données sur le PIB des pays, issues de la source World Wide Data.

Sauf précision contraire, les données sont exploitées ici dans leur
totalité, depuis la création des Jeux Olympiques modernes (1896).
Suivant l'analyse, les données incomplètes seront traitées différemment.
Ces choix seront justifiés si nécessaire.

\subsection{Problématique}\label{problematique}

Les Jeux Olympics sont devenus le symbole de la force physique et
mentale de l'être humain. C'est l'occasion pour que les élites de chaque
pays se réunissent et montrent leurs spectaculaires performances. D'un
autre côté, ces rencontres sont un sujet intéressant pour les
scientifiques : Le dernier sciècle a prédit que l'humain atteindra
bientôt ses limites physiques, tandis que l'on voit toujours aujourd'hui
des sportifs continuer d'établir de nouveaux records.

Avec plusieurs centaines d'années d'existance des JO, l'humanité a
identifié quelques facteurs optimisant la performance physique.
L'historique des JO pourraient surement fournir une indication à la
question: Quels sont les facteurs qui influent/optimisent la performance
physiques dans les JO ?

On peut aussi se demander quelle est l'influence de la richesse d'un
pays et de sa situation sociale sur les performances aux Jeux
olympiques.

Dans cette étude, on analyse ces problématiques via les grandes axes
suivant : les caractéristiques physiques des participants, leur âge,les
performances de chaque équipe nationale, la richesse de chaque pays.
Certains axes nous mènent à analyser des phénomènes particuliers.
D'autres nous demandent une étude supplémentaire pour mieux les
comprendre.

\section{1. Premièr axe: Les caractéristiques physiques des athlètes
inflencent leur
performance}\label{premier-axe-les-caracteristiques-physiques-des-athletes-inflencent-leur-performance}

\subsection{1.1 l'âge optimise la
performance}\label{lage-optimise-la-performance}

L'approche naïve permet ici de présenterle nombre de médailles que
chaque tranche d'âge a pu obtenir, quelque soit l'année des Jeux. La
performance est mesurée ici en fonction du nombre de médailles gagnées
ainsi qu'en fonction de la ``couleur'' de la médaille. C'est en effet un
bon indicteur pour discriminer les performances entre les athlètes sur
le podium et ceux n'ayant pas remporté de médaille.

\includegraphics{report_files/figure-latex/qual_medal_age-1.pdf}

D'après le résultat, on constate que l'âge entre 20 et 27 sera le pique
de maturité des sportifs. Cependant, cette induction est faite sur
l'hypothèse où le nombre de participant ne joue pas en jeu. En effet, il
peut arriver que l'on a beaucoup de participant à l'âge de 20 à 27, mais
très peu entre eux ont gagné, alors que peut-être ceux qui sont plus
âgés participe moins mais gagne plus.

Pour vérifier notre hypothèse, on calcule la ``Performance'' des
participant. L'indice de performance sera naïvement basé sur la
probabilité d'un participant figurant dans telle catégorie a la chance
de gagner un médaille ou pas. Formellement:

\(Performance = \frac{#Medal}{#Participant} = \frac{\sum(Gold + Silver + Bronze)}{\sum(all)}\)

\subsection{1.2 La taille entre-t-elle en question
?}\label{la-taille-entre-t-elle-en-question}

A travers l'étude suivante, on veut essayer de comprendre si la taille
des athlètes est corrélées avec leur performance. On présente ici les
médailles gagnées par les athlètes féminines et masculins en fonction de
leur taille.

\includegraphics{report_files/figure-latex/medals_height-1.pdf} Comme on
pouvait s'y attendre, le nombre de médailles suivant la taille a une
allure de distribussion gaussienne. La taille des hommes où le pic du
nombre de médailles gagnées est le plus important est à 1m80 pour 941
médailles. Chez les femmes, le pic est à 1m70, pour 535 médailles. Ces
tailles correspondent bien à des tailles normales pour des hommes et des
femmes. D'autre part, on visualise, par l'allure des courbes par sexe,
que les femmes sont généralement plus petites que les hommes

\section{2.2eme axe : la situation économique et sociale, un impact sur
les résultats
nationaux}\label{eme-axe-la-situation-economique-et-sociale-un-impact-sur-les-resultats-nationaux}

\subsection{2.1 Présentation}\label{presentation}

Premièrement, on peut noter que les Femmes ne sont autorisées à
participer aux JO qu'à partir de la 7ème édition, en 1920. Jusqu'alors,
seuls des hommes concourraient. D'autre part, on constate que les
éditions devant se dérouler pendant les guerres mondiales ont été
annulées. C'est le cas en 1916, 1940 et 1944.

A partir de ces 1eres constatations dans notre jeu de données, on peut
se demander si les conflits qu'a rencontré un pays, a pu se constater
dans une dégradation des résultats de l'édition à ce moment là.

\subsection{2.2 La croissance fulgurante des
femmes}\label{la-croissance-fulgurante-des-femmes}

Dans notre société actuelle, on aborde la place de la femme dans tous
les domaines. On aborde surtout le problème de la différence de
rémunération. Dans le graphe suivant, on peut observer l'évolution du
nombre de médailles remportés respectivement par les hommes et les
femmes, au cours des sessions olympiques.

\includegraphics{report_files/figure-latex/women territory-1.pdf}

On retrouve l'absence de femmes aux JO avant les années 1920. Mais
ensuite, on remarque surtout la progression réalisée en terme de nombre
de médailles gagnées par les femmes. Cela correspond à une évolution
positive de la place de la femme notre société. Une femme pouvant
participer et s'affirmer dans une compétition sportive est un signe de
liberté et d'égalité. On notera aussi l'augmentation croissante de la
proportion de femmes participants parmis les athlètes. De 3\% en 1920,
les femmes représentent 18\% des athlètes en 1960, 39\% en 2000. En
2016, elles représentaient 45\% de la population d'athèles.

\subsection{2.2 Analyse}\label{analyse}

\subsection{2.3 Résultats}\label{resultats}

\section{3.Problématique 3}\label{problematique-3}

\subsection{3.1 Présentation}\label{presentation-1}

\subsection{3.2 Analyse}\label{analyse-1}

\subsection{3.3 Résultats}\label{resultats-1}

\section{Conclusion}\label{conclusion}

\section{References}\label{references}


\end{document}
